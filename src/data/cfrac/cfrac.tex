\documentclass{amsart}
\usepackage{hyperref}
\usepackage{fullpage}
\usepackage{amsrefs}
\usepackage{tikz}
\usetikzlibrary{matrix,arrows,decorations.pathmorphing, cd}

\newcommand{\ceil}	{\operatorname{ceil}}
\newcommand{\frc}	{\operatorname{frac}}
\newcommand{\lst}	{\operatorname{list}}

\newcommand{\N}         {{\mathbb{N}}}
\newcommand{\Q}         {{\mathbb{Q}}}
\newcommand{\R}         {{\mathbb{R}}}
\newcommand{\Z}         {{\mathbb{Z}}}

\newcommand{\al}        {\alpha}
\newcommand{\bt}        {\beta} 
\newcommand{\tht}       {\theta}
\newcommand{\sg}        {\sigma}

\newcommand{\bbm}       {\left[\begin{matrix}}
\newcommand{\ebm}       {\end{matrix}\right]}
\newcommand{\bsm}       {\left[\begin{smallmatrix}}
\newcommand{\esm}       {\end{smallmatrix}\right]}

\newcommand{\bu}	{\bullet}
\newcommand{\st}        {\;|\;}
\newcommand{\tm}        {\times}

\renewcommand{\:}{\colon}

\newtheorem{theorem}{Theorem}[section]
\newtheorem{conjecture}[theorem]{Conjecture}
\newtheorem{lemma}[theorem]{Lemma}
\newtheorem{proposition}[theorem]{Proposition}
\newtheorem{corollary}[theorem]{Corollary}
\theoremstyle{definition}
\newtheorem{remark}[theorem]{Remark}
\newtheorem{definition}[theorem]{Definition}
\newtheorem{example}[theorem]{Example}
\newtheorem{construction}[theorem]{Construction}

\newtheorem{notation}{Notation}
\renewcommand{\thenotation}{} % make the notation environment unnumbered

%\numberwithin{equation}{subsection}

\begin{document}
\title{Number types and continued fractions}
\author{N.~P.~Strickland}

\maketitle 

This is a sketch of an approach to defining $\N$, $\Z$, $\Q$ and $\R$
in Lean or a similar system.

We define a \emph{quasiring} to be a structure with commutative and
associative rules for addition and multiplication, with a
multiplicative unit, such that multiplication distributes over
addition.  If $R$ is a quasiring and $a,b\in R$, we declare that $a<b$
iff there exists $x$ with $a+x=b$.  We say that $R$ is \emph{positive}
if this gives a total order on $R$.  We say that $R$ is a
\emph{quasifield} if every element has a multiplicative inverse.

If $R$ is a positive quasiring then one can check that $R\amalg\{0\}$
is a semiring and $R\amalg\{0\}\amalg(-R)$ is a ring.

We start by defining $\N^+$ in Peano style, and checking that it is a
positive quasiring.  We do not need to do anything about division at
this stage (but truncated subtraction will appear as part of the
analysis of the order).  This gives $\N=\N^+\amalg\{0\}$ as a
semiring, and $\Z=\N^+\amalg\{0\}\amalg(-\N^+)$ as a ring.

We now define $\Q^+_f$ to be the set of unreduced fractions $a/b$ with
$a,b\in\N^+$.  This is a quasiring under the operations of unreduced
addition and multiplication of fractions.  It has an inversion
operation $(a/b)^{-1}=(b/a)$, which satisfies $1^{-1}=1$ and
$(xy)^{-1}=x^{-1}y^{-1}$ and $(x^{-1})^{-1}=x$ but not $xx^{-1}=1$.

This quasiring is not positive, because trichotomy fails for the
canonical order relation.  We introduce an alternative preorder by
declaring that $a/b\preceq c/d$ iff $ad\leq bc$ in $\N^+$, and check
that this is compatible with the algebraic operations.  We then
introduce an equivalence relation by declaring that $x\sim y$ iff
$x\preceq y\preceq x$.  We put $\Q^+_r=\Q^+_f/\sim$ and check that
this becomes a positive quasifield.

Now define $M_w$ to be the free monoid generated by $S$ and $T$.  This
acts on $\Q^+_f$ by $S\bu(a/b)=((a+b)/b)$ and $T\bu(a/b)=(a/(a+b))$.
We define $\tht\:M_w\to\Q^+_f$ by $\tht(U)=U.(1/1)$.

Now define 
\[ M_m = \left\{\bbm a & b \\ c & d \ebm \in M_2(\N) \st 
                 ad = bc+1,\; d<b \text{ or } b=c=0. \right\}.
\]
One checks that for $\bsm a&b\\c&d\esm\in M_m$ we also have $c<a$.
Using this, we see that $M_m$ is closed under multiplication.  We can
define a monoid homomorphism $\phi\:M_w\to M_m$ by 
\[ \phi(S) = \bbm 1&1 \\ 0&1 \ebm \qquad 
   \phi(T) = \bbm 1&0 \\ 1&1 \ebm.
\]
This is actually an isomorphism, but it is probably best to postpone
the proof.  The monoid $M_m$ acts on $\Q^+_f$ by 
\[ \bbm a & b \\ c & d \ebm \bu \frac{u}{v}  =
    \frac{au+bv}{cu+dv},
\]
and the map $\phi$ is compatible with this.

Now define $\sg\:M_w\tm\Q^+_f\to M_w\tm\Q^+_f$ by 
\[ \sg(U,a/b) = \begin{cases}
    (U,a/b) & \text{ if } a = b \\
    (US,(a-b)/b) & \text{ if } a > b \\
    (UT,a/(b-a)) & \text{ if } a < b.
   \end{cases}
\]
This has the property that if $\sg(U,x)=(V,y)$ then $U\bu x=V\bu y$.
Also, if $x\sim x'$ and $\sg(U,x)=(V,y)$ then $\sg(U,x')=(V,y')$ with
$y\sim y'$.

Next, by induction on $a+b$ we see that $\sg^m(a/b)$ is independent of
$m$ for $m\gg 0$.  We write
$\sg^{\infty}(x)=(\psi(x),\rho(x)/\rho(x))$ for the eventual value of
$\sg^m(x)$.

We now put $\Q^+_c=\{x\in\Q^+_f\st\rho(x)=1\}$.  We find that
$\rho(\tht(U))=1$ and $\psi(\tht(U))=U$ so $\tht$ gives a map
$M_w\to\Q^+_c$.  We also find that $\tht(\psi(x))\in\Q^+_c$ and that
$x\sim\tht(\psi(x))$.  We therefore have a canonical equivalence
between $\Q^+_c$ and $\Q^+_f$, which we can use to transport the
quasiring structure from $\Q^+_r$ to $\Q^+_c$.  Alternatively, we can
define $\Q^+_m$ to be a copy of $M_m$.  The maps $\tht$ and $\psi$
give an equivalence from $\Q^+_m$ to $\Q^+_c$, and we can use this to
transport the quasiring structure to $\Q^+_m$.  

We can now define $\Q=\Q^+\amalg\{0\}\amalg(-\Q^+)$, using any of the
above models for $\Q^+$.  We may also want to define
$\Q_\infty=\Q\cup\{\infty\}$. 

We can define a ceiling map $\ceil\:\Q_f^+\to\N^+$ by 
\begin{align*}
 \ceil(a/b) &= \begin{cases}
  1 & \text{ if } a \leq b \\
  1 + \ceil((a-b)/b) & \text{ if } a > b.
 \end{cases}
\end{align*}
The corresponding map $M_w\to\N^+$ sends $S^k$ and $S^kTU$ to $k+1$.
There is a floor map $\Q^+_f\to\N$ and a fraction map
$\frc\:\Q^+_f\to\Q^+_f$ defined similarly.  We can use these to define
division with remainder on $\N^+$ and then obtain the ring structure
on $\Z/n$.

We can now define a Dedekind-type model $\R^+_d$ for $\R^+$.  An
element is a subset $u\subset\Q^+$ such that both $u$ and $u^c$ are
nonempty, $u$ is closed downwards, and $u$ has no largest element.
This becomes a positive quasifield with operations
\begin{align*}
 1   &= \{x\in\Q^+\st x<1\} \\
 u+v &= \{x+y\st x\in u,\;y\in v\} \\
 uv  &= \{xy\st x\in u,\;y\in v\}.
\end{align*}
I think that only trichotomy of the order should require classical
logic.  

We can now define a maps $\al,\bt\:\lst\N\to\Q_+$ by 
\begin{align*}
 \al(n_1,\dotsc,n_r) &= S^{n_1}TS^{n_2}T\dotsb TS^{n_r}0 \\
 \bt(n_1,\dotsc,n_r) &= S^{n_1}TS^{n_2}T\dotsb TS^{n_r}1.
\end{align*}
We find that $\al(u)\leq\al(uv)\leq\bt(uv)\leq\bt(u)$ for all lists
$u$ and $v$.  

Now put $\R^+_c=\{u\:\N\to\N\st \exists i, u(i)>0\}$, and order this
lexicographically.  For $u\in\R^+_c$ we put 
\[ \al(u) = \{x\in\Q^+ \st x<\al(u_{<r}) \text{ for some } r\in\N\}.
\]
One can check that this gives a strictly increasing map
$\al\:\R^+_c\to\R^+_d$.  With classical logic we can prove that it is
bijective.  

We can now define $\R=\R^+\amalg\{0\}\amalg(-\R^+)$.



\end{document}
