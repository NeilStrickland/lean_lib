\documentclass[9pt]{beamer}
\usepackage{beamerthemesplit}
\usetheme[headheight=0pt,footheight=0pt]{boxes}

\input xypic
\newdir{ >}{{}*!/-9pt/\dir{>}}

\usepackage{tikz}

\newcommand{\Nil}       {\operatorname{Nil}}

\newcommand{\R}         {{\mathbb{R}}}
\newcommand{\Z}         {{\mathbb{Z}}}
\newcommand{\N}         {{\mathbb{N}}}


\begin{document}

\begin{frame}[t]
 \frametitle{Commutative rings and nilpotents}

 \begin{itemize}
  \item A commutative ring is a set $R$ equipped with elements $0,1$ and
   rules for addition, multiplication, satisfying all the usual rules
   including $ab=ba$.
  \item Examples: integers $\Z$, real numbers $\R$, modular numbers
   $\Z/n$, polynomials $\Z[x]$.  Not $\N$ (no subtraction) or
   $M_n(\R)$ (matrices; $ab\neq ba$).
  \item An element $x\in R$ is
   \emph{nilpotent} if $x^{n+1}=0$ for some $n\geq 0$.  Example:
   $20^3$ is divisible by $1000$, so $20$ is nilpotent in $\Z/1000$.
   We define $\Nil(R)$ to be the set of nilpotent elements, so
   $\Nil(\Z/1000)=\{0,10,20,\dotsc,990\}$. 
  \item If $x$ and $y$ are nilpotent then so is $x+y$.  Example: if
   $x^3=y^4=0$ then
   \begin{align*}
    (x+y)^6 &=
      x^6 + 6x^5y + 15x^4y^2 + 20 x^3y^3 + 15x^2y^4 + 6xy^5 + y^6 \\
      &= (x^3+6x^2y+15xy^2+20y^3)x^3 + (15x^2+6xy+y^2)y^4 \\
      &= (\dotsb)0 + (\dotsb)0 = 0
   \end{align*}
  \item Easier facts: $0$ is nilpotent, and if $x$ is nilpotent, then
   so is $xy$.
  \item An \emph{ideal} is a subset $I\subseteq R$ such that
   (i)~$0\in I$ and~(ii) when $x,y\in I$ we have $x+y\in I$ and (iii)~
   when $x\in I$ and $y\in R$ we have $xy\in I$.  Easy example: the
   even numbers form an ideal in $\Z$.
  \item The above facts show that $\Nil(R)$ is an ideal in $R$.
 \end{itemize}
\end{frame}

\end{document}
